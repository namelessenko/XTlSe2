\documentclass[a4paper,12pt]{report}
\setcounter{secnumdepth}{4}
\setcounter{tocdepth}{3}
\usepackage[utf8]{inputenc}

%Language&Font
%\setmainfont{Times New Roman}
\usepackage{lmodern} 
\usepackage[T1]{fontenc}
\usepackage[english]{babel}
%========================================
%Pictures
\usepackage{pgfplots}
\usepackage{graphicx}
\graphicspath{{img/}}
\DeclareGraphicsExtensions{.pdf,.png,.jpg,.eps}
%========================================
%Title
\title{TlXSe2 Report}
\author{Nosenko A.}
\date{March 2022}
%========================================
%Table contents

%========================================

\begin{document}
\maketitle
\tableofcontents
\chapter{Computational details}

All computations have been performed using the Vienna Ab Initio Software Package (VASP) version 5.4.4.pl2. PBE parametrization of GGA functional was used within a projector-augmented-wave scheme taken from the standard VASP PAW library.

There are some problems with the calculation of f-electrons. This is not a VASP problem, this is a DFT problem in general. Therefore, the potential in which the f-electrons are included in the core was chosen for the calculation

The density of states for both materials was performed with cutoff energy set to 700 eV with Gamma-centred k-points mesh 28x28x28. Band calculation using spin-orbit coupling has been performed in the next k-path $\Gamma-T-H_2|H_0-L-\Gamma-S_0|S_2-F-\Gamma$ with 20 intersections.

Post-processing produced with pymatgen python library.
 
\chapter{Results}
\include{Er_res.tex}
\include{Tm_res.tex}
\end{document}